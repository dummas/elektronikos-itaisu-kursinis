\documentclass[11pt,a4paper]{article}
\usepackage[left=20mm,right=15mm,top=15mm,bottom=15mm]{geometry}
\usepackage[utf8x]{inputenc}
\usepackage[L7x]{fontenc}
\usepackage[lithuanian]{babel}
\begin{document}
\begin{titlepage}
  
  \begin{center}
    \textsc{\LARGE Vilniaus Gedimino Technikos universitetas}\\[2mm]
    \textsc{\Large Elektroninių sistemų katedra}\\[70mm]
    \textsc{\Large Terminės difuzijos proceso tyrimas, tranzistorių ekvivalentinių grandinių schemų sudarymas ir akustoelektroninio įtaiso projektavimas}\\[10mm]
    \textsc{\normalsize Elektronikos įtaisų kursinio darbo užduotis}\\[40mm]
    \begin{minipage}{1\textwidth}
      \begin{flushright}
        \emph{Darbą atliko:} EI-08/2 gr. studentas\\ Maksim Norkin\\
        \emph{Darbą tikrino:} V.Malinauskas\\
      \end{flushright}
    \end{minipage}
    \vfill
    {\large Vilnius \\ \the\year}
  \end{center}
\end{titlepage}
\tableofcontents
\newpage
\section{Įvadas}
\section{Difuzijos proceso tyrimas}
\subsection{Apskaičiuoti ir nubraižyti priemaišų pasiskirstymą po priemaišų įterpimo etapo}
Pradiniai duomenys:\\
Difuzijos koeficienas ($cm^2/s$): $D = 1.8*10^{-13}$\\
Proceso trukmės: $t_1 = 13min = 780s, t_2 = 18min = 1080s$ ir $t_3 = 22min = 1320s$.\\


\end{document}
