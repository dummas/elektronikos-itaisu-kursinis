\documentclass[11pt,a4paper]{article}
\usepackage[left=20mm,right=15mm,top=15mm,bottom=15mm]{geometry}
\usepackage[utf8x]{inputenc}
\usepackage[L7x]{fontenc}
\usepackage[lithuanian]{babel}
\usepackage{tikz}
\usepackage{pgfplots}
\pgfplotsset{compat=newest}
\begin{document}
\begin{titlepage}
  
  \begin{center}
    \textsc{\LARGE Vilniaus Gedimino Technikos universitetas}\\[2mm]
    \textsc{\Large Elektroninių sistemų katedra}\\[70mm]
    \textsc{\Large Terminės difuzijos proceso tyrimas, tranzistorių ekvivalentinių grandinių schemų sudarymas ir akustoelektroninio įtaiso projektavimas}\\[10mm]
    \textsc{\normalsize Elektronikos įtaisų kursinio darbo užduotis}\\[40mm]
    \begin{minipage}{1\textwidth}
      \begin{flushright}
        \emph{Darbą atliko:} EI-08/2 gr. studentas\\ Maksim Norkin\\
        \emph{Darbą tikrino:} V.Malinauskas\\
      \end{flushright}
    \end{minipage}
    \vfill
    {\large Vilnius \\ \the\year}
  \end{center}
\end{titlepage}
\tableofcontents
\newpage
\section{Įvadas}
\section{Difuzijos proceso tyrimas}
\subsection{Apskaičiuoti ir nubraižyti priemaišų pasiskirstymą po priemaišų įterpimo etapo}
Pradiniai duomenys:\\
Difuzijos koeficienas ($cm^2/s$): $D = 1.8*10^{-13}$\\
Proceso trukmės: $t_1 = 13min = 780s, t_2 = 18min = 1080s$ ir $t_3 = 22min = 1320s$.\\
Difuzija iš nesenkančio šaltinio vyksta pirmojoje - priemaišų įterpimo stadijoje. Jos pasisskirstymas randamas, pasitelkus:
\[N(x,t) = N_0*erfc \left( \frac{x}{x\sqrt{Dt}} \right) \] \\
\begin{tikzpicture}[domain=0:0.005]
  \begin{axis} [
      xlabel=$x$,
      ylabel=$N(x)/N_0$,
      grid=major]
    \addplot[smooth, mark=*] plot function {erfc(x/(2*sqrt(1.8*10**(-13)*780)))};
    \addlegendentry{$t_1$}
    \addplot[smooth, mark=o] plot function {erfc(x/(2*sqrt(1.8*10**(-13)*1080)))};
    \addlegendentry{$t_2$}
    \addplot[smooth, mark=x] plot function {erfc(x/(2*sqrt(1.8*10**(-13)*1320)))};
    \addlegendentry{$t_3$}
  \end{axis}
\end{tikzpicture}

\subsection{Apskaičiuoti ir nubraižyti, kaip priemaišų įterpimo etape kinta priemaišos srauto tankis ir legiravimo dozė}
Pradiniai duomenys:\\
Priemaišų koncentracija bandinio paviršiuje ($1/cm^2$): $N_0 = 3*10^{20}$\\
Priemaišos srauto tankis išreiškiamas:\\
\[J(0,t) = N_0*\sqrt{\frac{D}{\pi t}}\]
\begin{tikzpicture}[domain=0:1]
  \begin{axis} [
      xlabel=$t$,
      ylabel=$J(0\,t)$,
      grid=major
    ]
    \addplot[smooth] plot function {3*10**(20)*sqrt(1.8*10**(-13)/(3.14*x))};
  \end{axis}
\end{tikzpicture} \\

Legiravimo dozė išreiškiama:\\
\[Q(t) = 2*N_0*\sqrt{\frac{Dt}{\pi}}\]
\begin{tikzpicture}[domain=0:1]
  \begin{axis} [
      xlabel=$t$,
      ylabel=$Q(t)$,
      grid=major
    ]
    \addplot[smooth] plot function {3*10**(20)*sqrt((1.8*10**(-13)*x)/(3.14))};
  \end{axis}
\end{tikzpicture} \\
\subsection{Apskaičiuoti ir nubraižyti priemaišų pasiskirstymą po priemaišų perskirstymo etapo}
Pradiniai duomenys:\\
Legiravimo dozė ($1/cm^3$): $Q = 1.4*10^{13}$;\\
Difuzijos koeficientas ($cm^2/S$): $D = 1.8*10^{-13}$;\\
Proceso trukmės: $t_1 = 81min = 4860s$; $t_2 = 70min = 4200s$; $t_3 = 21min = 1260s$;\\
Priemaišų perskirstymo stadiją galime apibrėžti:\\
\[N(x,t') = \frac{Q}{\sqrt{\pi D' t'}} exp \left( - \frac{x^2}{4D't'} \right)  \]

\begin{tikzpicture}[domain=0:0.005]
  \begin{axis} [
      ylabel=$N(x)$,
      xlabel=$x$,
      grid=major
    ]
    % Q/sqrt(pi*D'*t')*exp(-(x^2)/(4*D't'))
    \addplot[smooth,mark=*] plot function{ (1.4*10**(13)/(sqrt(pi*1.8*10**(-13)))*exp(-(x**2)/(4*1.8*10**(-13)*4860))) };
    \addlegendentry{$t_1$}
    \addplot[smooth,mark=o] plot function{ (1.4*10**(13)/(sqrt(pi*1.8*10**(-13)))*exp(-(x**2)/(4*1.8*10**(-13)*4200))) };
    \addlegendentry{$t_2$}
    \addplot[smooth,mark=x] plot function{ (1.4*10**(13)/(sqrt(pi*1.8*10**(-13)))*exp(-(x**2)/(4*1.8*10**(-13)*1260))) };
    \addlegendentry{$t_3$}
  \end{axis}
\end{tikzpicture}

\subsection{Apskaičiuoti ir nubraižyti priemaišų pasiskirstymą tranzistoriuje, formuojamame dvikartės difuzijos būdu}

\begin{tabular}{|r|c|c|c|}\hline
  & Įterpimo stadija & Perskirstymo stadija & Įterpimo stadija \\ \hline
  Priemaišų koncentracija ($1/cm^3$): & $2.5*10^{19}$ & - & $1.4*10^{21}$ \\ \hline
  Normali temperatūra ($^{\circ}C$): & $1000$ & $1000$ & $1000$ \\ \hline
  Aktyvacijos energija ($eV$): & $2.2$ & $3.7$ & $2.9$ \\ \hline
  Difuzijos koeficientas ($cm^2/s$): & $1.9*10^{-13}$ & $1.4*10^{-13}$ & $3.8*10^{-13}$ \\  \hline
  Stadijos trukmė ($min.$): & $24$ & $77$ & $42$ \\ \hline
  Faktinė temperatūra ($^{\circ}C$): & $942$ & $1056$ & $318$ \\ \hline
\end{tabular}
% Dvikartė difuzija
% Bazes difuzija




\section{Dvipolio tranzistoriaus parametrų skaičiavimas ir ekvivalentinės grandinės schemos sudarymas}
Pradiniai duomenys:\\
$I_B = 0.14 mA$, $U_{CE} = 8V$, $f_{T} = 0.9 GHz$;\\
\subsection{Rasti dvipolio tranzistoriaus $h$ parametrus}
\subsection{Sudaryti tranzistoriaus $\Pi$ pavidalo ekvivalentinės grandinės schemą, rasti jos elementų parametrus}
\subsection{Apskaičiuoti išėjimo srovės kintamąją dedamąją, kai kintamoji įėjimo įtampa yra 115mV}
\subsection{Rasti žemo dažnio įtampos stiprinimo koeficiantą, kai apkrovos varža lygi 892 $\Omega$ }

\section{Lauko tranzistoriaus parametrų skaičiavimas ir ekvivalentinės grandinės schemos sudarymas}
\subsection{Nubraižyti lauko tranzistoriaus perdavimo charkteristikas, kai $U_{DS} = 3,7,11 V$}
\subsection{Apskaičiuoti lauko tranzistoriaus parametrus nurodytame darbo taške}
\subsection{Sudaryti lauko tranzistoriaus ekvivalentinės grandinės schemą}
\subsection{Apskaičiuoti $f_{T}$, kai $C_{11}$ = 4, $C_{12} = 9$ pF}
\subsection{Apskaičiuoti kintamąją išėjimo srovės dedamąją, kai kintamosios įėjimo įtampos amplitudė yra 135 mV}
\subsection{Rasti žemo dažnio įtampos stiprinimo koeficientą, kai apkrovos varža lygi 621 $\Omega$}

\section{Akustinės elektronikos įtaiso projektavimas}
Pradiniai duomenys:\\
Centrinis pralaidumo juostos dažnis (MHz): 100\\
Pralaidumo juostos plotis (MHz): 8

\end{document}
