\documentclass[11pt,a4paper]{article}
\usepackage[left=20mm,right=15mm,top=15mm,bottom=15mm]{geometry}
\usepackage[utf8x]{inputenc}
\usepackage[L7x]{fontenc}
\usepackage[lithuanian]{babel}
\usepackage{tikz}
\usepackage{pgfplots}
\pgfplotsset{compat=newest}
\begin{document}
\begin{titlepage}
  
  \begin{center}
    \textsc{\LARGE Vilniaus Gedimino Technikos universitetas}\\[2mm]
    \textsc{\Large Elektroninių sistemų katedra}\\[70mm]
    \textsc{\Large Terminės difuzijos proceso tyrimas, tranzistorių ekvivalentinių grandinių schemų sudarymas ir akustoelektroninio įtaiso projektavimas}\\[10mm]
    \textsc{\normalsize Elektronikos įtaisų kursinio darbo užduotis}\\[40mm]
    \begin{minipage}{1\textwidth}
      \begin{flushright}
        \emph{Darbą atliko:} EI-08/2 gr. studentas\\ Maksim Norkin\\
        \emph{Darbą tikrino:} V.Malinauskas\\
      \end{flushright}
    \end{minipage}
    \vfill
    {\large Vilnius \\ \the\year}
  \end{center}
\end{titlepage}
\tableofcontents
\newpage
\section{Įvadas}
\section{Difuzijos proceso tyrimas}
\subsection{Apskaičiuoti ir nubraižyti priemaišų pasiskirstymą po priemaišų įterpimo etapo}
Pradiniai duomenys:\\
Difuzijos koeficienas ($cm^2/s$): $D = 1.8*10^{-13}$\\
Proceso trukmės: $t_1 = 13min = 780s, t_2 = 18min = 1080s$ ir $t_3 = 22min = 1320s$.\\
Difuzija iš nesenkančio šaltinio vyksta pirmojoje - priemaišų įterpimo stadijoje. Jos pasisskirstymas randamas, pasitelkus:
\[N(x,t) = N_0*erfc \left( \frac{x}{x\sqrt{Dt}} \right) \] \\
\begin{tikzpicture}[domain=0:0.005]
  \begin{axis} [
      xlabel=$x$,
      ylabel=$N(x)/N_0$,
      grid=major]
    \addplot[smooth, mark=*] plot function {erfc(x/(2*sqrt(1.8*10**(-13)*780)))};
    \addlegendentry{$t_1$}
    \addplot[smooth, mark=o] plot function {erfc(x/(2*sqrt(1.8*10**(-13)*1080)))};
    \addlegendentry{$t_2$}
    \addplot[smooth, mark=x] plot function {erfc(x/(2*sqrt(1.8*10**(-13)*1320)))};
    \addlegendentry{$t_3$}
  \end{axis}
\end{tikzpicture}

\subsection{Apskaičiuoti ir nubraižyti, kaip priemaišų įterpimo etape kinta priemaišos srauto tankis ir legiravimo dozė}
Pradiniai duomenys:\\
Priemaišų koncentracija bandinio paviršiuje ($1/cm^2$): $N_0 = 3*10^{20}$\\
Priemaišos srauto tankis išreiškiamas:\\
\[J(0,t) = N_0*\sqrt{\frac{D}{\pi t}}\]
\begin{tikzpicture}[domain=0:1]
  \begin{axis} [
      xlabel=$t$,
      ylabel=$J(0\,t)$,
      grid=major
    ]
    \addplot[smooth] plot function {3*10**(20)*sqrt(1.8*10**(-13)/(3.14*x))};
  \end{axis}
\end{tikzpicture} \\

Legiravimo dozė išreiškiama:\\
\[Q(t) = 2*N_0*\sqrt{\frac{Dt}{\pi}}\]
\begin{tikzpicture}[domain=0:1]
  \begin{axis} [
      xlabel=$t$,
      ylabel=$Q(t)$,
      grid=major
    ]
    \addplot[smooth] plot function {3*10**(20)*sqrt((1.8*10**(-13)*x)/(3.14))};
  \end{axis}
\end{tikzpicture} \\
\subsection{Apskaičiuoti ir nubraižyti priemaišų pasiskirstymą po priemaišų perskirstymo etapo}
Pradiniai duomenys:\\
Legiravimo dozė ($1/cm^3$): $Q = 1.4*10^{13}$;\\
Difuzijos koeficientas ($cm^2/S$): $D = 1.8*10^{-13}$;\\
Procesotrukmės: $t_1 = 81min = 4860s$; $t_2 = 70min = 4200s$; $t_3 = 21min = 1260s$;\\

\end{document}
